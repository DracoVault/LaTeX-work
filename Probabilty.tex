\documentclass{article}
\usepackage{graphicx} % Required for inserting images
\usepackage{amsmath}

\title{probability}
\author{Dhananjay singh}
\date{April 2023}

\begin{document}

\maketitle

\section*{Problems}
\begin{itemize}
 \item 1) A probability of a man hitting a target is $1/3$.

a) if he fires  6 times , what is a probabilty of hitting almost 5 times.

b) at least 5 times

c) exactly 1 time

2) if he fires , so that probability of him hitting target once is greater than $3/4$. find the value of n.



\item 2) Assume that on the average 1 telephone number out of 15 is called between 2pm and 3 pm on weak day is busy . what is a probability if 6 random selected number are called.
a)not more than 3 
b) at least 3 of them will be busy

\item 3) Fit a binomial data to a  follow data when tossing 5 coins.

x : 0  1  2  3  4  5 


f(x): 2 14 20 34 22 8

\item 4) 10 coins are tossed 124 times and following frequency are observed . Calculate the expected frequency

no of head: 0 1 2 3 4 5 6 7 8 9 10

frequency: 2 10 38 106 188 257 226 128 59 7 3

\item 5) In binomial distribution consist of 5 independent trials . probability of 1 \& 2 success are 0.4096 and 0.2048. Find p of the following distribution.

\end{itemize}


\title{\Large poisson distribution}

\begin{itemize}
    

\item 6) Number of telephone calls arrive on internal switch board of an office is $90/hrs$.  Probability  that almost 1 to 3 calls in a min at a board arrives.

\item 7) Suppose a book of 585 pages contain 43 typographical error . if these errors are randomly selected throughout the book. what is the  Probability  that 10 pages is selected at random will be free at random.

\item 8) Average number of accident on any day on a national highway is $1.8$. Determine the Probability that number of accident are 



i) at least once 

ii) at most once 

\item 9) In a certain factory turning out razer blade , there is a small chance of 0.002
for any blade to be defected . the blade are supply in packet of 10. Calculate the approximate number of packets.


i) contain no defect

ii) 1 defect 

iii) 2 defect in a  consignment of 1000 packets.

\item 10) an insurance company has discovered that only 0.1\% of the population is involved in certain type of accident every year.

i) probabilty that not more than 5 of its client are in such accident next year.

\item 11) If a random variable follows a possion distribution such that $p(x=2) = 9.p(x=4)+90.p(x=6).$ Find mean and variance.

\item 12) A car higher furn has 2 cars which it hire out day by day. The number of demands for a  car each day is possion distribution with mean = 1.5 . Calculate the proportion of days on 

i) which neither car is used 

ii) some demand is refused.

\item 13) For a possion distribution on a following 

$X:$ 0 1 2 3 4 5 

$f(x):$ 192 100 24 3 1

To calculate the mean .
\end{itemize}


\title{\Large chebyshev's inequality}

\begin{itemize}

    
    \item 14) If a random variable X such that expected value of $[x]$ = 3 & $E[x^2]$ = 13. Use the chebyshev's inequality to determine the lower bound for $P[ -2 < X < 8]$.if X is a variate such that E[x]= 3 and $E[x^2]$ = 13. Show that $P[-2 < X < 8 \geq 21/5$ . 

    \item 15)Variate X takes value -1, 1,3 ,5 with associated probabilty 
    $1/6, 1/6, 1/6, 1/2$. Compute $P[|X-3|] \geq 1$ directly & found on upper bound to P by using chebyshev's inequality.

    \item 16) If X is a number scored in a fair die . Show that chebyshev's inequality give $P([X-u] > 2.5) < 0.47$.

   Such that $1 \leq X \leq 6$.

   \item 15) A random variable X can assume only the values , [x = 1,2,3,4,. . . . ] with probabilty 2 . Show that chebyshev's inequality gives $P[(x-4)-2] \geq 1/2$. While the actual probabilty is 15/16. 

   \item  16) The rainfall of X in certain locality is normally distributed random variable with mean 10mm/hr and var is 4mm/hr. Find the simple upper bound on probabity that rainfall in partial year will exceed the mean by 5mm/hr.
\end{itemize}

\end{document}
